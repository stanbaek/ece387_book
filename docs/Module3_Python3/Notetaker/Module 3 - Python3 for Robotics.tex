\documentclass{handout}


\SetCourseTitle{ECE495: Fundamentals of Robotics Research}
\SetHandoutTitle{Module 3 - Python3 for Robotics}
%\SetDueDate{8 Sep at 0715 (via Gradescope)}
%\ShowAllBlanks

%\showsoln \setsolncolor{red}

\newlist{todolist}{itemize}{2}
\setlist[todolist]{label=$\square$}
\usepackage{pifont}
\newcommand{\cmark}{\ding{51}}%
\newcommand{\xmark}{\ding{55}}%
\newcommand{\done}{\rlap{$\square$}{\raisebox{2pt}{\large\hspace{1pt}\cmark}}%
	\hspace{-2.5pt}}
\newcommand{\wontfix}{\rlap{$\square$}{\large\hspace{1pt}\xmark}}

\usepackage{hyperref}

\definecolor{code}{HTML}{ECF8F4}
\definecolor{comments}{HTML}{5269A5}

\usepackage[T1]{fontenc}
\lstset{%
	language=bash, upquote=true,
	otherkeywords={rostopic, rosnode, rosrun, roscore, cd, ls, sudo, nano, echo, mkdir, touch, chmod, catkin\_make, rosmsg, rosservice, catkin\_create\_pkg, rospack, ssh, rosed},
	showspaces=false, showtabs=false, showstringspaces=false, upquote=true, tabsize=4,
	literate={~} {$\sim$}{1},
	showstringspaces=false,
	xleftmargin=0.06\textwidth,
	linewidth=0.99\textwidth,
	columns=fullflexible,
	backgroundcolor=\color{code},
	keepspaces=true,
	breaklines=true,
	basicstyle={\small\fontfamily{fvm}\fontseries{m}\selectfont},
	keywordstyle={\small\fontfamily{fvm}\fontseries{b}\selectfont},
	commentstyle={\color{comments}\small\fontfamily{fvm}\itshape\selectfont},
	belowcaptionskip=10pt,
	float=h
}

\graphicspath{{./figs/}}

\begin{document}
\maketitle

\begin{figure}[H]
	\centering
	\includegraphics[width=.75\textwidth]{Cover.PNG}
\end{figure}

\textbf{Lesson Objectives:}
\begin{enumerate} \setlength\itemsep{0em}
	\item Learn fundamental concepts of Python
	\item Learn basic syntax of Python
	\item Understand Object Oriented Programming
	\item Develop basic operational understanding of Python through application
\end{enumerate}

\textbf{Agenda:}
\begin{enumerate} \setlength\itemsep{0em}
	\item Python3 Jupyter Notebook.
	\item ICE3 Jupyter Notebook.
\end{enumerate}

\newpage
\clearpage
\pagebreak

\section{Python3 Jupyter Notebook.}
We will use a Jupyter Notebook to practice and provide a Python3 refresher.

\begin{enumerate}
	\item On the master, open the Jupyter Notebook server:
	
\begin{lstlisting}[language=bash]
dfec@master:~$ roscd usafabot_curriculum/Module3_Python3
dfec@master:~$ jupyter-notebook
\end{lstlisting}
	
	\item Open the Python3 Jupyter Notebook, "Module3\_Python3.ipynb", and follow the instructions within the notebook.
	
\end{enumerate}

\section{ICE3 Jupyter Notebook.}
The ICE3 Jupyter Notebook will guide you through implementation of a chat client/server using ROS and Python3.

\begin{enumerate}\setlength\itemsep{1em}
	\item Ensure roscore is terminated (ctrl+c) before moving on to the ICE.
	\item On the master, open the Jupyter Notebook server (if it is not already open):
\begin{lstlisting}[language=bash]
dfec@master:~$ roscd usafabot_curriculum/Module3_Python3
dfec@master:~$ jupyter-notebook
\end{lstlisting}
	
	\item Open the ICE3 Jupyter Notebook, "ICE3\_Client.ipynb" and follow the instructions within the notebook. 
\end{enumerate}

\textbf{Checkpoint. Take a screenshot or show the instructor the following:}
\begin{enumerate}
	\item The output of each of the code blocks within the "ICE3\_ROS.ipynb" notebook.
\end{enumerate}

\section{Assignments.}
	\begin{todolist}
		\item Complete Jupyter Notebooks if not accomplished during class.
	\end{todolist}

\section{Next time.}
	\begin{itemize}
		\item \textbf{Lesson 8} - Quiz and ICE 3
		\item \textbf{Lesson 9} - Quiz and Module 4 - Driving the Robot
	\end{itemize}

\end{document}
